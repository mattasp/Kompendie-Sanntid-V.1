\chapter{QNX overview}

compliance: POSIX 1003.1-2001
Unix, threads, timers, signals

Arkitektur
Microkernel
- OS consist of a small kernel and a set of cooperating processes
- OS består av en liten kernel og et set cooprative(samarbeidende) prosesser.
- Prosessene er seperate fra kjernen(kernelen) så hvis noe går galt i en prosess vil dette ikke på virke kjernen.

Fordeler med denne typen arkitektur
- Stabilitet og rubusthet
- Enkelt å konfigurere og re-konfigurere

Ulemper:
- mer overhead
    - more context switches
    - more copies of data(data/memory is not shared)
    
QNX microkernel

Mikrokernelen i QNX har 6 fundamentale tjenester(services) for operativsystem funksjoner.
- Synkronisasjon
- Scheduler
- Tråder(Threading)
- Timere(Timers)
- IPC(Inter-process communication)
- Interrupt redirectur(Forstyrrelse vidresender)

Microkernel services:
1.native IPC
2.thread support
3.thread scheduling
4.thread synchronization
5.timers
6.interrupt redirection

Mikrokernelen tar imot hardware interrupts og vidresender den til prosesser.

/* redundat information */
Process:
- Program loaded into memory
- identified with a pid(process id)
- owns spesific resources such as:
    - Memory, code and data
    - open files
    - identity
    - timers
    - mutexes
    - channels
    - (Could be more such as connection ?)

resources owned by one process are protected from other processes
some resources can be owned by more than one process.

Thread:
Single flow of execution/control
attributes:
* priority
* Scheduling alg.
* register set
* CPU mask for SMP(Symmetric multprocessing)
* signal mask
* others?

All attributes has to do with running code:

Threads in processes, 
A process must have at least one thread
threads in a process share all the process resources.
Threads run code
Processes owns resources.

IPC (Inter-Process Communication):
- OS processes and your processes cooperate using IPC. Together the OS and your processes makes up one seamless sysstem.
- A lot of different types of IPCs

Kernel IPC
- Messages
    - Exhcange information between processes
- Pulses
    - Delivirng notification to a process
- Signals
    - Interrupting a process and making it do something different(Usually used for termination)
    
QNX Native IPC API:
MsgSend()-->MsgReceive()->process msg
MsgSend()<--MsgReply()

Other IPC methods in IPC includes
    -built on top of native IPC
        - pipes
        - POSIX message queues
        - TPC/IP Sockets
        


Keywords:
Process
Thread
CPU(Central processing unit)
MuTex(Mutal Exclusive)
SMP(Symmetric multiprocessing)

